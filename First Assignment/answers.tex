\documentclass[a4paper]{article}

\usepackage[english]{babel}
\usepackage[utf8]{inputenc}
\usepackage{hyperref}

\title{Applied Cloud Computing
Uppsala University - Autum 2015
Report for Assignment 1}

\author{John Shaw}
\date{\today}

\begin{document}

\maketitle

\section*{Task 1}

\begin{enumerate}
    \item \textit{What is the difference between the private IP and the floating IP?}
    
    The private IP is local in the cloud network. The floating IP is connected and transfering the connection to the the associated private IP. The floating IP is an external IP accessable through Internet.
    \item \textit{Can you access the Internet from the VM without assigning a floating IP to the machine?}
    
    Only if I'm on the private network which I suspect I will never be.
    \item \textit{What is the difference between image, instance and snapshot?}
    
    An Image is static data containing the software while an Instance is a running virtual machine that's started from an Image.
    To create a new snapshot is to create a new image based on a already running instance.
    \item \textit{What is the name of the OpenStack service responsible for providing the: a. Image Service b. Compute Service}
    
        \begin{enumerate}
            \item \textit{Image Service}
            
            Glance
            \item \textit{Compute Service}
            
            Nova
        \end{enumerate}
\end{enumerate}

\section*{Task 2}

\begin{enumerate}
    \item \textit{What is the technology used to provide volumes in OpenStack? Is it RAID or LVM?}
    
    LVM
    \item \textit{What is LVM? Explain the advantage(s) of using LVM?}
    
    Logical Volume Manager provides an abstract layer ontop of the pysical disks, making the pysical disks not exposed to the OS. It uses block storages.
    
    One great advantage of using LVM is that the user can view the abstract volumes in a dynamic way. Making it possible to resize the volume without restarting the system. The user don't need to worry about the pysical disk(s). 
    \item \textit{Can one volume be attached to multiple instances or vice versa?}
    
    After testing I found out that many volumes can attach to one instance, not the opposite.
    \item \textit{Explain the main difference between Ephemeral Storage and Block-Storage. What are the major use-cases for the different storage types?}
    
    From a users point of view an Empheral Storage will ``dissapear'' when the VM is terminated. This is what happends when only Nova is deployed.
    
    A block storage is persistant and can therefore be detatched from one VM and attached to another. 
    Major usecase would be block storage for persistant data and Empheral Storage for non persistant data.
    \item \textit{Does your VM have ephemeral storage?}
    
    Yes, the data that's not on the volume I created is in Empheral storage.
\end{enumerate}

\section*{Task 3}

\begin{enumerate}
    \item \textit{Explain the picture in the tab ``Network Topology''}
    
    It's the external network that is connected to the internal network through a router on 192.168.0.254.
    \item \textit{What if the subnet used by the Tenant?}
    
    192.168.0.0/24
    \item \textit{What is the role of the router?}
    
    I'm assuming it directs the trafic between the external and internal network.
    \item \textit{Explain the path of the traffic of the VM to the Internet?}
    
    The path starts by connecting through the router and then into Internet.
    \item \textit{Find out the unique ID of the external network.}
    
    ext-net
    \item \textit{What is the name of the OpenStack service handling Networks?}
    
    Neutron
\end{enumerate}

\section*{Task 4}
\url{http://smog.uppmax.uu.se:8080/swift/v1/Lab1}
\begin{enumerate}
    \item \textit{Explain the difference between a folder on your UNIX filesystem and a pseudo-folder inside a container?}
    
    Things stored in a container is written to several disks in the network to ensure data replication. Objects in a container have an access point for the external network, making it possible to access the contents without SSH or similar teqniues that the UNIX filesystem would require. 
    \item \textit{The corresponding system in Amazon Web Services is called "S3". Is there a principal difference between an "S3 bucket" and a container in OpenStack's object store?}
    \item \textit{What is the name of the OpenStack service providing the Object Store?}
    
    Swift
\end{enumerate}

\end{document}